\documentclass{article}
\usepackage{graphicx}
\usepackage[margin = 2cm]{geometry}
\usepackage{caption}
\usepackage{subcaption}

\title{PCO Laboratoire 6 \\
\large Producteur-Consommateur pour calcul différé (Hoare)}
\author{Benoît Delay, Eva Ray}

\setlength{\parskip}{1em}

\begin{document}
\maketitle

\section{Etape 1}
Dans cette étape, nous avons mis en place la distribution des calculs grâce aux méthodes \texttt{requestComputation} et
\texttt{getWork}.
\subsection{Choix d'implémentation}
Pour représenter le buffer contenant les calculs à réaliser, nous avons choisi d'utiliser une map ayant comme clé le
type du calcul et comme valeur une \texttt{std::list} contenant les calculs sous forme de \texttt{Request} à réaliser de
ce type.
Notre
technique consiste simplement à ajouter les calculs au début de la liste correspondante dans la map dès que
\texttt{requestComputation} est appelée.
Cependant, chaque liste a une taille maximale définie par la constante \texttt{MAX\_TOLERATED\_QUEUE\_SIZE}. Pour
modéliser cette contrainte, nous avons créé un tableau nommé \texttt{fullQueuePerType} contenant une \texttt{Condition}
pour chaque type de calcul.
Avant d'ajouter le calcul à la liste, on vérifie donc que la taille de la liste est inférieure à \texttt{MAX\_TOLERATED\_QUEUE\_SIZE}.
Si ce n'est pas les cas, on appelle la fonction \texttt{wait} sur la condition correspondante dans \texttt{fullQueuePerType
} et on attend qu'un calcul du bon type arrive dans le buffer.
La méthode \texttt{requestComputation} est aussi en charge de donner un indice unique à chaque calcul. Pour cela,
nous avons utilisé un attribut statique \texttt{nextId} qui est incrémenté à chaque fois qu'un calcul est ajouté au
buffer.


La méthode \texttt{getWork}, quant à elle, demande du travail d'un certain type. Pour cela, elle vérifie que la liste
correspondante dans la map n'est pas vide. Si c'est le cas, elle appelle la méthode \texttt{pop\_back} sur la liste
afin de récupérer le dernier élément de la liste, qui sera donc le plus ancien. Si la liste est vide, elle appelle la
méthode \texttt{wait} sur la condition correspondante dans \texttt{emptyQueuePerType} qui est un autre tableau
contenant une \texttt{Condition} pour chaque type de calcul qui représente le fait qu'il n'y a pas de calcul de ce
type dans le buffer. Dans ce cas, on attented donc qu'un calcul de ce type soit ajouté au buffer avant de pouvoir
continuer.


Lorsqu'un calcul est ajouté au buffer dans la méthode \texttt{requestComputation}, on signale sur la condition correspondante
dans \texttt{emptyQueuePerType} que le buffer contient maintenant un calcul de ce type. De même, lorsqu'un calcul est
retiré du buffer dans la méthode \texttt{getWork}, on signale sur la condition correspondante dans \texttt{fullQueuePerType}
que le buffer contient maintenant un calcul de ce type en moins.
\subsection{Tests}

\section{Etape 2}
Dans cette étape nous avons mis en place la gestion des résultats grâce aux méthodes \texttt{getNextResult} et
\texttt{provideResult}.
\subsection{Choix d'implémentation}
Pour représenter la structure contenant les résultats, nous avons choisi d'utiliser une \texttt{std::list} contenant les
résultats sous forme de \texttt{ResultWithId}. \texttt{ResultWithId} est une structure que nous avons rajoutée
contenant un id et un résultat qui est optionel (grâce à \texttt{std::optional}). Ainis, nous pouvons ajouter le
résultat dans la liste directement après avoir envoyé la requête au calculateur, même lorsque le résultat est encore
en cours de calcul, ce qui nous permet de connaître l'ordre des calculs en cours ou terminé à tout moment. Dans les
faits, un résultat est ajouté à la liste dès que \texttt{requestComputation} est appelée.

La méthode \texttt{getNextResult} est en charge de récupérer le résultat le plus ancien dans la liste. Pour cela, elle
n'as donc qu'à accéder au dernier élément contenu dans la liste. Si la partie optionelle du résultat est vide, cela
signifie que le résultat n'est pas encore disponible et donc que le calcul est toujours en cours. Dans ce cas, on
appelle la méthode \texttt{wait} sur la condition \texttt{notExpectedResult}. Cette dernière est une condition qui
représente le fait que le prochain résultat n'est pas encore disponible. On attend donc que le résultats attendu
soit disponible dans la liste avant de pouvoir continuer. Lorsque c'est le cas, on retire le résultat de la liste et
on le retourne.

La méthode \texttt{provideResult} permet au calculateur de retourner le résultat du calcul. Pour cela, elle cherche
l'id du résultat donné dans la liste \texttt{results} et met à jour la partie optionelle du résultat avec le résultat
donné. Ensuite, elle signale sur la condition \texttt{notExpectedResult} que le prochain résultat est disponible.


\subsection{Tests}

\section{Etape 3}
Dans cette étape, nous avons mis en place la possibilité d’annuler des calculs demandés par le client grâce aux méthodes
\texttt{abortComputation} et \texttt{continueWork}.
\subsection{Choix d'implémentation}
La méthode \texttt{abortComputation} permet d’annuler un calcul en cours grâce à son identifiant. Il y a plusieurs
cas à gérer. Si le calcul n’est pas encore en cours, il faut simplement le retirer de la map \texttt{buffer}. Dans
ce cas, il faut signaler sur la condition \texttt{fullQueuePerType} que la queue pour ce type de calcul contient un
calcul en moins. Si le calcul est en cours, il faut le retirer de la liste \texttt{results} et signaler sur
la condition \texttt{notExpectedResult} pour potentiellement débloqué un thread qui attend sur ce résultat.
Si le calcul est terminé, il faut simplement le retirer de la liste \texttt{results}.
On notera qu'au vu de notre implémentation de \texttt{requestComputation}, qui ajoute directement un calcul demandé
dans la liste \texttt{results}, il faut aussi retirer le calcul de cette liste s'il est demandé mais pas encore en
cours.

La méthode \texttt{continueWork} et quant à elle assez simple, puisqu'elle va simplement chercher l'id passé en
paramètre dans la liste \texttt{results}. Si l'id est présent, cela signifie que le calcul doit continuer, sinon, cela
signifie que le calcul doit être annulé.

\subsection{Tests}

\section{Etape 4}

\subsection{Choix d'implémentation}

\subsection{Tests}

\end{document}