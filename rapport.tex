\documentclass{article}
\usepackage{graphicx}
\usepackage[margin = 2cm]{geometry}
\usepackage{caption}
\usepackage{subcaption}

\title{PCO Laboratoire 6 \\
\large Producteur-Consommateur pour calcul différé (Hoare)}
\author{Benoît Delay, Eva Ray}

\setlength{\parskip}{1em}

\begin{document}
\maketitle

\section{Etape 1}
Dans cette étape, nous avons mis en place la distribution des calculs grâce aux méthodes \texttt{requestComputation} et
\texttt{getWork}.
\subsection{Choix d'implémentation}
Pour représenter le buffer contenant les calculs à réaliser, nous avons choisi d'utiliser une map ayant comme clé le
type du calcul et comme valeur une \texttt{std::list} contenant les \texttt{Computation} à réaliser de ce type. Notre
technique consiste simplement à ajouter les calculs au début de la liste correspondante dans la map dès que
\texttt{requestComputation} est appelée.
Cependant, chaque liste a une taille maximale définie par la constante \texttt{MAX\_TOLERATED\_QUEUE\_SIZE}. Pour
modéliser cette contrainte, nous avons créé un tableau nommé \texttt{fullQueuePerType} contenant une \texttt{Condition}
pour chaque type de calcul.
Avant d'ajouter le calcul à la liste, on vérifie donc que la taille de la liste est inférieure à \texttt{MAX\_TOLERATED\_QUEUE\_SIZE}.
Si ce n'est pas les cas, on appelle la fonction \texttt{wait} sur la condition correspondante dans \texttt{fullQueuePerType
} et on attend qu'un calcul du bon type arrive dans le buffer.
La méthode \texttt{requestComputation} est aussi en charge de donner un indice unique à chaque calcul. Pour cela,
nous avons utilisé un attribut statique \texttt{nextId} qui est incrémenté à chaque fois qu'un calcul est ajouté au
buffer.


La méthode \texttt{getWork}, quant à elle, demande du travail d'un certain type. Pour cela, elle vérifie que la liste
correspondante dans la map n'est pas vide. Si c'est le cas, elle appelle la méthode \texttt{pop\_back} sur la liste
afin de récupérer le dernier élément de la liste, qui sera donc le plus ancien. Si la liste est vide, elle appelle la
méthode \texttt{wait} sur la condition correspondante dans \texttt{emptyQueuePerType} qui est un autre tableau
contenant une \texttt{Condition} pour chaque type de calcul qui représente le fait qu'il n'y a pas de calcul de ce
type dans le buffer. Dans ce cas, on attented donc qu'un calcul de ce type soit ajouté au buffer avant de pouvoir
continuer.


Lorsqu'un calcul est ajouté au buffer dans la méthode \texttt{requestComputation}, on signale sur la condition correspondante
dans \texttt{emptyQueuePerType} que le buffer contient maintenant un calcul de ce type. De même, lorsqu'un calcul est
retiré du buffer dans la méthode \texttt{getWork}, on signale sur la condition correspondante dans \texttt{fullQueuePerType}
que le buffer contient maintenant un calcul de ce type en moins.
\subsection{Tests}

\section{Etape 2}
\subsection{Choix d'implémentation}
\subsection{Tests}

\section{Etape 3}
\subsection{Choix d'implémentation}
\subsection{Tests}

\section{Etape 4}
\subsection{Choix d'implémentation}
\subsection{Tests}

\end{document}